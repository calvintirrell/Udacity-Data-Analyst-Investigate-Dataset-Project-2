
% Default to the notebook output style

    


% Inherit from the specified cell style.




    
\documentclass[11pt]{article}

    
    
    \usepackage[T1]{fontenc}
    % Nicer default font (+ math font) than Computer Modern for most use cases
    \usepackage{mathpazo}

    % Basic figure setup, for now with no caption control since it's done
    % automatically by Pandoc (which extracts ![](path) syntax from Markdown).
    \usepackage{graphicx}
    % We will generate all images so they have a width \maxwidth. This means
    % that they will get their normal width if they fit onto the page, but
    % are scaled down if they would overflow the margins.
    \makeatletter
    \def\maxwidth{\ifdim\Gin@nat@width>\linewidth\linewidth
    \else\Gin@nat@width\fi}
    \makeatother
    \let\Oldincludegraphics\includegraphics
    % Set max figure width to be 80% of text width, for now hardcoded.
    \renewcommand{\includegraphics}[1]{\Oldincludegraphics[width=.8\maxwidth]{#1}}
    % Ensure that by default, figures have no caption (until we provide a
    % proper Figure object with a Caption API and a way to capture that
    % in the conversion process - todo).
    \usepackage{caption}
    \DeclareCaptionLabelFormat{nolabel}{}
    \captionsetup{labelformat=nolabel}

    \usepackage{adjustbox} % Used to constrain images to a maximum size 
    \usepackage{xcolor} % Allow colors to be defined
    \usepackage{enumerate} % Needed for markdown enumerations to work
    \usepackage{geometry} % Used to adjust the document margins
    \usepackage{amsmath} % Equations
    \usepackage{amssymb} % Equations
    \usepackage{textcomp} % defines textquotesingle
    % Hack from http://tex.stackexchange.com/a/47451/13684:
    \AtBeginDocument{%
        \def\PYZsq{\textquotesingle}% Upright quotes in Pygmentized code
    }
    \usepackage{upquote} % Upright quotes for verbatim code
    \usepackage{eurosym} % defines \euro
    \usepackage[mathletters]{ucs} % Extended unicode (utf-8) support
    \usepackage[utf8x]{inputenc} % Allow utf-8 characters in the tex document
    \usepackage{fancyvrb} % verbatim replacement that allows latex
    \usepackage{grffile} % extends the file name processing of package graphics 
                         % to support a larger range 
    % The hyperref package gives us a pdf with properly built
    % internal navigation ('pdf bookmarks' for the table of contents,
    % internal cross-reference links, web links for URLs, etc.)
    \usepackage{hyperref}
    \usepackage{longtable} % longtable support required by pandoc >1.10
    \usepackage{booktabs}  % table support for pandoc > 1.12.2
    \usepackage[inline]{enumitem} % IRkernel/repr support (it uses the enumerate* environment)
    \usepackage[normalem]{ulem} % ulem is needed to support strikethroughs (\sout)
                                % normalem makes italics be italics, not underlines
    

    
    
    % Colors for the hyperref package
    \definecolor{urlcolor}{rgb}{0,.145,.698}
    \definecolor{linkcolor}{rgb}{.71,0.21,0.01}
    \definecolor{citecolor}{rgb}{.12,.54,.11}

    % ANSI colors
    \definecolor{ansi-black}{HTML}{3E424D}
    \definecolor{ansi-black-intense}{HTML}{282C36}
    \definecolor{ansi-red}{HTML}{E75C58}
    \definecolor{ansi-red-intense}{HTML}{B22B31}
    \definecolor{ansi-green}{HTML}{00A250}
    \definecolor{ansi-green-intense}{HTML}{007427}
    \definecolor{ansi-yellow}{HTML}{DDB62B}
    \definecolor{ansi-yellow-intense}{HTML}{B27D12}
    \definecolor{ansi-blue}{HTML}{208FFB}
    \definecolor{ansi-blue-intense}{HTML}{0065CA}
    \definecolor{ansi-magenta}{HTML}{D160C4}
    \definecolor{ansi-magenta-intense}{HTML}{A03196}
    \definecolor{ansi-cyan}{HTML}{60C6C8}
    \definecolor{ansi-cyan-intense}{HTML}{258F8F}
    \definecolor{ansi-white}{HTML}{C5C1B4}
    \definecolor{ansi-white-intense}{HTML}{A1A6B2}

    % commands and environments needed by pandoc snippets
    % extracted from the output of `pandoc -s`
    \providecommand{\tightlist}{%
      \setlength{\itemsep}{0pt}\setlength{\parskip}{0pt}}
    \DefineVerbatimEnvironment{Highlighting}{Verbatim}{commandchars=\\\{\}}
    % Add ',fontsize=\small' for more characters per line
    \newenvironment{Shaded}{}{}
    \newcommand{\KeywordTok}[1]{\textcolor[rgb]{0.00,0.44,0.13}{\textbf{{#1}}}}
    \newcommand{\DataTypeTok}[1]{\textcolor[rgb]{0.56,0.13,0.00}{{#1}}}
    \newcommand{\DecValTok}[1]{\textcolor[rgb]{0.25,0.63,0.44}{{#1}}}
    \newcommand{\BaseNTok}[1]{\textcolor[rgb]{0.25,0.63,0.44}{{#1}}}
    \newcommand{\FloatTok}[1]{\textcolor[rgb]{0.25,0.63,0.44}{{#1}}}
    \newcommand{\CharTok}[1]{\textcolor[rgb]{0.25,0.44,0.63}{{#1}}}
    \newcommand{\StringTok}[1]{\textcolor[rgb]{0.25,0.44,0.63}{{#1}}}
    \newcommand{\CommentTok}[1]{\textcolor[rgb]{0.38,0.63,0.69}{\textit{{#1}}}}
    \newcommand{\OtherTok}[1]{\textcolor[rgb]{0.00,0.44,0.13}{{#1}}}
    \newcommand{\AlertTok}[1]{\textcolor[rgb]{1.00,0.00,0.00}{\textbf{{#1}}}}
    \newcommand{\FunctionTok}[1]{\textcolor[rgb]{0.02,0.16,0.49}{{#1}}}
    \newcommand{\RegionMarkerTok}[1]{{#1}}
    \newcommand{\ErrorTok}[1]{\textcolor[rgb]{1.00,0.00,0.00}{\textbf{{#1}}}}
    \newcommand{\NormalTok}[1]{{#1}}
    
    % Additional commands for more recent versions of Pandoc
    \newcommand{\ConstantTok}[1]{\textcolor[rgb]{0.53,0.00,0.00}{{#1}}}
    \newcommand{\SpecialCharTok}[1]{\textcolor[rgb]{0.25,0.44,0.63}{{#1}}}
    \newcommand{\VerbatimStringTok}[1]{\textcolor[rgb]{0.25,0.44,0.63}{{#1}}}
    \newcommand{\SpecialStringTok}[1]{\textcolor[rgb]{0.73,0.40,0.53}{{#1}}}
    \newcommand{\ImportTok}[1]{{#1}}
    \newcommand{\DocumentationTok}[1]{\textcolor[rgb]{0.73,0.13,0.13}{\textit{{#1}}}}
    \newcommand{\AnnotationTok}[1]{\textcolor[rgb]{0.38,0.63,0.69}{\textbf{\textit{{#1}}}}}
    \newcommand{\CommentVarTok}[1]{\textcolor[rgb]{0.38,0.63,0.69}{\textbf{\textit{{#1}}}}}
    \newcommand{\VariableTok}[1]{\textcolor[rgb]{0.10,0.09,0.49}{{#1}}}
    \newcommand{\ControlFlowTok}[1]{\textcolor[rgb]{0.00,0.44,0.13}{\textbf{{#1}}}}
    \newcommand{\OperatorTok}[1]{\textcolor[rgb]{0.40,0.40,0.40}{{#1}}}
    \newcommand{\BuiltInTok}[1]{{#1}}
    \newcommand{\ExtensionTok}[1]{{#1}}
    \newcommand{\PreprocessorTok}[1]{\textcolor[rgb]{0.74,0.48,0.00}{{#1}}}
    \newcommand{\AttributeTok}[1]{\textcolor[rgb]{0.49,0.56,0.16}{{#1}}}
    \newcommand{\InformationTok}[1]{\textcolor[rgb]{0.38,0.63,0.69}{\textbf{\textit{{#1}}}}}
    \newcommand{\WarningTok}[1]{\textcolor[rgb]{0.38,0.63,0.69}{\textbf{\textit{{#1}}}}}
    
    
    % Define a nice break command that doesn't care if a line doesn't already
    % exist.
    \def\br{\hspace*{\fill} \\* }
    % Math Jax compatability definitions
    \def\gt{>}
    \def\lt{<}
    % Document parameters
    \title{No Show Patient Appointments}
    
    
    

    % Pygments definitions
    
\makeatletter
\def\PY@reset{\let\PY@it=\relax \let\PY@bf=\relax%
    \let\PY@ul=\relax \let\PY@tc=\relax%
    \let\PY@bc=\relax \let\PY@ff=\relax}
\def\PY@tok#1{\csname PY@tok@#1\endcsname}
\def\PY@toks#1+{\ifx\relax#1\empty\else%
    \PY@tok{#1}\expandafter\PY@toks\fi}
\def\PY@do#1{\PY@bc{\PY@tc{\PY@ul{%
    \PY@it{\PY@bf{\PY@ff{#1}}}}}}}
\def\PY#1#2{\PY@reset\PY@toks#1+\relax+\PY@do{#2}}

\expandafter\def\csname PY@tok@w\endcsname{\def\PY@tc##1{\textcolor[rgb]{0.73,0.73,0.73}{##1}}}
\expandafter\def\csname PY@tok@c\endcsname{\let\PY@it=\textit\def\PY@tc##1{\textcolor[rgb]{0.25,0.50,0.50}{##1}}}
\expandafter\def\csname PY@tok@cp\endcsname{\def\PY@tc##1{\textcolor[rgb]{0.74,0.48,0.00}{##1}}}
\expandafter\def\csname PY@tok@k\endcsname{\let\PY@bf=\textbf\def\PY@tc##1{\textcolor[rgb]{0.00,0.50,0.00}{##1}}}
\expandafter\def\csname PY@tok@kp\endcsname{\def\PY@tc##1{\textcolor[rgb]{0.00,0.50,0.00}{##1}}}
\expandafter\def\csname PY@tok@kt\endcsname{\def\PY@tc##1{\textcolor[rgb]{0.69,0.00,0.25}{##1}}}
\expandafter\def\csname PY@tok@o\endcsname{\def\PY@tc##1{\textcolor[rgb]{0.40,0.40,0.40}{##1}}}
\expandafter\def\csname PY@tok@ow\endcsname{\let\PY@bf=\textbf\def\PY@tc##1{\textcolor[rgb]{0.67,0.13,1.00}{##1}}}
\expandafter\def\csname PY@tok@nb\endcsname{\def\PY@tc##1{\textcolor[rgb]{0.00,0.50,0.00}{##1}}}
\expandafter\def\csname PY@tok@nf\endcsname{\def\PY@tc##1{\textcolor[rgb]{0.00,0.00,1.00}{##1}}}
\expandafter\def\csname PY@tok@nc\endcsname{\let\PY@bf=\textbf\def\PY@tc##1{\textcolor[rgb]{0.00,0.00,1.00}{##1}}}
\expandafter\def\csname PY@tok@nn\endcsname{\let\PY@bf=\textbf\def\PY@tc##1{\textcolor[rgb]{0.00,0.00,1.00}{##1}}}
\expandafter\def\csname PY@tok@ne\endcsname{\let\PY@bf=\textbf\def\PY@tc##1{\textcolor[rgb]{0.82,0.25,0.23}{##1}}}
\expandafter\def\csname PY@tok@nv\endcsname{\def\PY@tc##1{\textcolor[rgb]{0.10,0.09,0.49}{##1}}}
\expandafter\def\csname PY@tok@no\endcsname{\def\PY@tc##1{\textcolor[rgb]{0.53,0.00,0.00}{##1}}}
\expandafter\def\csname PY@tok@nl\endcsname{\def\PY@tc##1{\textcolor[rgb]{0.63,0.63,0.00}{##1}}}
\expandafter\def\csname PY@tok@ni\endcsname{\let\PY@bf=\textbf\def\PY@tc##1{\textcolor[rgb]{0.60,0.60,0.60}{##1}}}
\expandafter\def\csname PY@tok@na\endcsname{\def\PY@tc##1{\textcolor[rgb]{0.49,0.56,0.16}{##1}}}
\expandafter\def\csname PY@tok@nt\endcsname{\let\PY@bf=\textbf\def\PY@tc##1{\textcolor[rgb]{0.00,0.50,0.00}{##1}}}
\expandafter\def\csname PY@tok@nd\endcsname{\def\PY@tc##1{\textcolor[rgb]{0.67,0.13,1.00}{##1}}}
\expandafter\def\csname PY@tok@s\endcsname{\def\PY@tc##1{\textcolor[rgb]{0.73,0.13,0.13}{##1}}}
\expandafter\def\csname PY@tok@sd\endcsname{\let\PY@it=\textit\def\PY@tc##1{\textcolor[rgb]{0.73,0.13,0.13}{##1}}}
\expandafter\def\csname PY@tok@si\endcsname{\let\PY@bf=\textbf\def\PY@tc##1{\textcolor[rgb]{0.73,0.40,0.53}{##1}}}
\expandafter\def\csname PY@tok@se\endcsname{\let\PY@bf=\textbf\def\PY@tc##1{\textcolor[rgb]{0.73,0.40,0.13}{##1}}}
\expandafter\def\csname PY@tok@sr\endcsname{\def\PY@tc##1{\textcolor[rgb]{0.73,0.40,0.53}{##1}}}
\expandafter\def\csname PY@tok@ss\endcsname{\def\PY@tc##1{\textcolor[rgb]{0.10,0.09,0.49}{##1}}}
\expandafter\def\csname PY@tok@sx\endcsname{\def\PY@tc##1{\textcolor[rgb]{0.00,0.50,0.00}{##1}}}
\expandafter\def\csname PY@tok@m\endcsname{\def\PY@tc##1{\textcolor[rgb]{0.40,0.40,0.40}{##1}}}
\expandafter\def\csname PY@tok@gh\endcsname{\let\PY@bf=\textbf\def\PY@tc##1{\textcolor[rgb]{0.00,0.00,0.50}{##1}}}
\expandafter\def\csname PY@tok@gu\endcsname{\let\PY@bf=\textbf\def\PY@tc##1{\textcolor[rgb]{0.50,0.00,0.50}{##1}}}
\expandafter\def\csname PY@tok@gd\endcsname{\def\PY@tc##1{\textcolor[rgb]{0.63,0.00,0.00}{##1}}}
\expandafter\def\csname PY@tok@gi\endcsname{\def\PY@tc##1{\textcolor[rgb]{0.00,0.63,0.00}{##1}}}
\expandafter\def\csname PY@tok@gr\endcsname{\def\PY@tc##1{\textcolor[rgb]{1.00,0.00,0.00}{##1}}}
\expandafter\def\csname PY@tok@ge\endcsname{\let\PY@it=\textit}
\expandafter\def\csname PY@tok@gs\endcsname{\let\PY@bf=\textbf}
\expandafter\def\csname PY@tok@gp\endcsname{\let\PY@bf=\textbf\def\PY@tc##1{\textcolor[rgb]{0.00,0.00,0.50}{##1}}}
\expandafter\def\csname PY@tok@go\endcsname{\def\PY@tc##1{\textcolor[rgb]{0.53,0.53,0.53}{##1}}}
\expandafter\def\csname PY@tok@gt\endcsname{\def\PY@tc##1{\textcolor[rgb]{0.00,0.27,0.87}{##1}}}
\expandafter\def\csname PY@tok@err\endcsname{\def\PY@bc##1{\setlength{\fboxsep}{0pt}\fcolorbox[rgb]{1.00,0.00,0.00}{1,1,1}{\strut ##1}}}
\expandafter\def\csname PY@tok@kc\endcsname{\let\PY@bf=\textbf\def\PY@tc##1{\textcolor[rgb]{0.00,0.50,0.00}{##1}}}
\expandafter\def\csname PY@tok@kd\endcsname{\let\PY@bf=\textbf\def\PY@tc##1{\textcolor[rgb]{0.00,0.50,0.00}{##1}}}
\expandafter\def\csname PY@tok@kn\endcsname{\let\PY@bf=\textbf\def\PY@tc##1{\textcolor[rgb]{0.00,0.50,0.00}{##1}}}
\expandafter\def\csname PY@tok@kr\endcsname{\let\PY@bf=\textbf\def\PY@tc##1{\textcolor[rgb]{0.00,0.50,0.00}{##1}}}
\expandafter\def\csname PY@tok@bp\endcsname{\def\PY@tc##1{\textcolor[rgb]{0.00,0.50,0.00}{##1}}}
\expandafter\def\csname PY@tok@fm\endcsname{\def\PY@tc##1{\textcolor[rgb]{0.00,0.00,1.00}{##1}}}
\expandafter\def\csname PY@tok@vc\endcsname{\def\PY@tc##1{\textcolor[rgb]{0.10,0.09,0.49}{##1}}}
\expandafter\def\csname PY@tok@vg\endcsname{\def\PY@tc##1{\textcolor[rgb]{0.10,0.09,0.49}{##1}}}
\expandafter\def\csname PY@tok@vi\endcsname{\def\PY@tc##1{\textcolor[rgb]{0.10,0.09,0.49}{##1}}}
\expandafter\def\csname PY@tok@vm\endcsname{\def\PY@tc##1{\textcolor[rgb]{0.10,0.09,0.49}{##1}}}
\expandafter\def\csname PY@tok@sa\endcsname{\def\PY@tc##1{\textcolor[rgb]{0.73,0.13,0.13}{##1}}}
\expandafter\def\csname PY@tok@sb\endcsname{\def\PY@tc##1{\textcolor[rgb]{0.73,0.13,0.13}{##1}}}
\expandafter\def\csname PY@tok@sc\endcsname{\def\PY@tc##1{\textcolor[rgb]{0.73,0.13,0.13}{##1}}}
\expandafter\def\csname PY@tok@dl\endcsname{\def\PY@tc##1{\textcolor[rgb]{0.73,0.13,0.13}{##1}}}
\expandafter\def\csname PY@tok@s2\endcsname{\def\PY@tc##1{\textcolor[rgb]{0.73,0.13,0.13}{##1}}}
\expandafter\def\csname PY@tok@sh\endcsname{\def\PY@tc##1{\textcolor[rgb]{0.73,0.13,0.13}{##1}}}
\expandafter\def\csname PY@tok@s1\endcsname{\def\PY@tc##1{\textcolor[rgb]{0.73,0.13,0.13}{##1}}}
\expandafter\def\csname PY@tok@mb\endcsname{\def\PY@tc##1{\textcolor[rgb]{0.40,0.40,0.40}{##1}}}
\expandafter\def\csname PY@tok@mf\endcsname{\def\PY@tc##1{\textcolor[rgb]{0.40,0.40,0.40}{##1}}}
\expandafter\def\csname PY@tok@mh\endcsname{\def\PY@tc##1{\textcolor[rgb]{0.40,0.40,0.40}{##1}}}
\expandafter\def\csname PY@tok@mi\endcsname{\def\PY@tc##1{\textcolor[rgb]{0.40,0.40,0.40}{##1}}}
\expandafter\def\csname PY@tok@il\endcsname{\def\PY@tc##1{\textcolor[rgb]{0.40,0.40,0.40}{##1}}}
\expandafter\def\csname PY@tok@mo\endcsname{\def\PY@tc##1{\textcolor[rgb]{0.40,0.40,0.40}{##1}}}
\expandafter\def\csname PY@tok@ch\endcsname{\let\PY@it=\textit\def\PY@tc##1{\textcolor[rgb]{0.25,0.50,0.50}{##1}}}
\expandafter\def\csname PY@tok@cm\endcsname{\let\PY@it=\textit\def\PY@tc##1{\textcolor[rgb]{0.25,0.50,0.50}{##1}}}
\expandafter\def\csname PY@tok@cpf\endcsname{\let\PY@it=\textit\def\PY@tc##1{\textcolor[rgb]{0.25,0.50,0.50}{##1}}}
\expandafter\def\csname PY@tok@c1\endcsname{\let\PY@it=\textit\def\PY@tc##1{\textcolor[rgb]{0.25,0.50,0.50}{##1}}}
\expandafter\def\csname PY@tok@cs\endcsname{\let\PY@it=\textit\def\PY@tc##1{\textcolor[rgb]{0.25,0.50,0.50}{##1}}}

\def\PYZbs{\char`\\}
\def\PYZus{\char`\_}
\def\PYZob{\char`\{}
\def\PYZcb{\char`\}}
\def\PYZca{\char`\^}
\def\PYZam{\char`\&}
\def\PYZlt{\char`\<}
\def\PYZgt{\char`\>}
\def\PYZsh{\char`\#}
\def\PYZpc{\char`\%}
\def\PYZdl{\char`\$}
\def\PYZhy{\char`\-}
\def\PYZsq{\char`\'}
\def\PYZdq{\char`\"}
\def\PYZti{\char`\~}
% for compatibility with earlier versions
\def\PYZat{@}
\def\PYZlb{[}
\def\PYZrb{]}
\makeatother


    % Exact colors from NB
    \definecolor{incolor}{rgb}{0.0, 0.0, 0.5}
    \definecolor{outcolor}{rgb}{0.545, 0.0, 0.0}



    
    % Prevent overflowing lines due to hard-to-break entities
    \sloppy 
    % Setup hyperref package
    \hypersetup{
      breaklinks=true,  % so long urls are correctly broken across lines
      colorlinks=true,
      urlcolor=urlcolor,
      linkcolor=linkcolor,
      citecolor=citecolor,
      }
    % Slightly bigger margins than the latex defaults
    
    \geometry{verbose,tmargin=1in,bmargin=1in,lmargin=1in,rmargin=1in}
    
    

    \begin{document}
    
    
    \maketitle
    
    

    
    \section{Project: Investigating Appointments With No-Show
Patients}\label{project-investigating-appointments-with-no-show-patients}

\subsection{Table of Contents}\label{table-of-contents}

Introduction

Data Wrangling

Exploratory Data Analysis

Conclusions

     \#\# Introduction

\begin{quote}
I will be exploring the No Show Patient Appointments dataset. I
downloaded the No Show Appointments dataset and saved it as
'noshow.csv'. Using the Udacity provided Jupyter Notebook template I
imported all the packages that might be needed. Some questions I look to
answer through analyzing this dataset include:

\begin{enumerate}
\def\labelenumi{\arabic{enumi})}
\item
  Does age (younger or older) have any kind of impact or relationship in
  terms of people showing up for appointments or not?
\item
  Does the combination of age and scholarship have some kind of impact
  or relationship in terms of people showing up for appointments or not?
\end{enumerate}
\end{quote}

\subsubsection{Import Packages}\label{import-packages}

    \begin{Verbatim}[commandchars=\\\{\}]
{\color{incolor}In [{\color{incolor}1}]:} \PY{k+kn}{import} \PY{n+nn}{pandas} \PY{k}{as} \PY{n+nn}{pd}
        \PY{k+kn}{import} \PY{n+nn}{numpy} \PY{k}{as} \PY{n+nn}{np}
        \PY{k+kn}{import} \PY{n+nn}{matplotlib}\PY{n+nn}{.}\PY{n+nn}{pyplot} \PY{k}{as} \PY{n+nn}{plt}
        \PY{k+kn}{import} \PY{n+nn}{seaborn} \PY{k}{as} \PY{n+nn}{sns}
        \PY{o}{\PYZpc{}} \PY{n}{matplotlib} \PY{n}{inline}
\end{Verbatim}


     \#\# Data Wrangling

\begin{quote}
My initial attempt to read the CSV file led to the following error:
``UnicodeDecodeError: 'utf-8' codec can't decode byte 0xed in position
3: invalid continuation byte''.
\end{quote}

\begin{quote}
A quick Google search led to me to this StackOverflow question and the
answer I needed to solve this issue.
https://stackoverflow.com/questions/5552555/unicodedecodeerror-invalid-continuation-byte
\end{quote}

\subsubsection{General Properties}\label{general-properties}

    \begin{Verbatim}[commandchars=\\\{\}]
{\color{incolor}In [{\color{incolor}2}]:} \PY{c+c1}{\PYZsh{} Load your data and save it to \PYZsq{}appt\PYZsq{} (short for appointments).}
        \PY{c+c1}{\PYZsh{} The appt dataframe is all the appointments, regardless of whether or not the person showed up or not.}
        \PY{n}{appt} \PY{o}{=} \PY{n}{pd}\PY{o}{.}\PY{n}{read\PYZus{}csv}\PY{p}{(}\PY{l+s+s1}{\PYZsq{}}\PY{l+s+s1}{noshow.csv}\PY{l+s+s1}{\PYZsq{}}\PY{p}{,} \PY{n}{encoding}\PY{o}{=}\PY{l+s+s1}{\PYZsq{}}\PY{l+s+s1}{latin\PYZhy{}1}\PY{l+s+s1}{\PYZsq{}}\PY{p}{)}
        
        \PY{c+c1}{\PYZsh{} Print out a few lines; appt.tail() will print out the last 5 lines from the dataset (if desired).}
        \PY{n}{appt}\PY{o}{.}\PY{n}{head}\PY{p}{(}\PY{p}{)}
        \PY{c+c1}{\PYZsh{} appt.tail()}
\end{Verbatim}


\begin{Verbatim}[commandchars=\\\{\}]
{\color{outcolor}Out[{\color{outcolor}2}]:}       PatientId  AppointmentID Gender          ScheduledDay  \textbackslash{}
        0  2.990000e+13        5642903      F  2016-04-29T18:38:08Z   
        1  5.590000e+14        5642503      M  2016-04-29T16:08:27Z   
        2  4.260000e+12        5642549      F  2016-04-29T16:19:04Z   
        3  8.680000e+11        5642828      F  2016-04-29T17:29:31Z   
        4  8.840000e+12        5642494      F  2016-04-29T16:07:23Z   
        
                 AppointmentDay  Age      Neighbourhood  Scholarship  Hipertension  \textbackslash{}
        0  2016-04-29T00:00:00Z   62    JARDIM DA PENHA            0             1   
        1  2016-04-29T00:00:00Z   56    JARDIM DA PENHA            0             0   
        2  2016-04-29T00:00:00Z   62      MATA DA PRAIA            0             0   
        3  2016-04-29T00:00:00Z    8  PONTAL DE CAMBURI            0             0   
        4  2016-04-29T00:00:00Z   56    JARDIM DA PENHA            0             1   
        
           Diabetes  Alcoholism  Handcap  SMS\_received No-show  
        0         0           0        0             0      No  
        1         0           0        0             0      No  
        2         0           0        0             0      No  
        3         0           0        0             0      No  
        4         1           0        0             0      No  
\end{Verbatim}
            
    \begin{Verbatim}[commandchars=\\\{\}]
{\color{incolor}In [{\color{incolor}3}]:} \PY{c+c1}{\PYZsh{} Perform operations to inspect data types and look for instances of missing or possibly errant data.}
        \PY{c+c1}{\PYZsh{} The describe function helps to get an overview from a statistical viewpoint.}
        \PY{n}{appt}\PY{o}{.}\PY{n}{describe}\PY{p}{(}\PY{p}{)}
\end{Verbatim}


\begin{Verbatim}[commandchars=\\\{\}]
{\color{outcolor}Out[{\color{outcolor}3}]:}           PatientId  AppointmentID            Age    Scholarship  \textbackslash{}
        count  1.105270e+05   1.105270e+05  110527.000000  110527.000000   
        mean   1.474961e+14   5.675305e+06      37.088874       0.098266   
        std    2.560943e+14   7.129575e+04      23.110205       0.297675   
        min    3.920000e+04   5.030230e+06      -1.000000       0.000000   
        25\%    4.170000e+12   5.640286e+06      18.000000       0.000000   
        50\%    3.170000e+13   5.680573e+06      37.000000       0.000000   
        75\%    9.440000e+13   5.725524e+06      55.000000       0.000000   
        max    1.000000e+15   5.790484e+06     115.000000       1.000000   
        
                Hipertension       Diabetes     Alcoholism        Handcap  \textbackslash{}
        count  110527.000000  110527.000000  110527.000000  110527.000000   
        mean        0.197246       0.071865       0.030400       0.022248   
        std         0.397921       0.258265       0.171686       0.161543   
        min         0.000000       0.000000       0.000000       0.000000   
        25\%         0.000000       0.000000       0.000000       0.000000   
        50\%         0.000000       0.000000       0.000000       0.000000   
        75\%         0.000000       0.000000       0.000000       0.000000   
        max         1.000000       1.000000       1.000000       4.000000   
        
                SMS\_received  
        count  110527.000000  
        mean        0.321026  
        std         0.466873  
        min         0.000000  
        25\%         0.000000  
        50\%         0.000000  
        75\%         1.000000  
        max         1.000000  
\end{Verbatim}
            
    \begin{Verbatim}[commandchars=\\\{\}]
{\color{incolor}In [{\color{incolor}4}]:} \PY{c+c1}{\PYZsh{} Perform operations to inspect data types and look for instances of missing or possibly errant data.}
        \PY{c+c1}{\PYZsh{} The info function helps to see the different data types for each columns as well as number of rows.}
        \PY{n}{appt}\PY{o}{.}\PY{n}{info}\PY{p}{(}\PY{p}{)}
\end{Verbatim}


    \begin{Verbatim}[commandchars=\\\{\}]
<class 'pandas.core.frame.DataFrame'>
RangeIndex: 110527 entries, 0 to 110526
Data columns (total 14 columns):
PatientId         110527 non-null float64
AppointmentID     110527 non-null int64
Gender            110527 non-null object
ScheduledDay      110527 non-null object
AppointmentDay    110527 non-null object
Age               110527 non-null int64
Neighbourhood     110527 non-null object
Scholarship       110527 non-null int64
Hipertension      110527 non-null int64
Diabetes          110527 non-null int64
Alcoholism        110527 non-null int64
Handcap           110527 non-null int64
SMS\_received      110527 non-null int64
No-show           110527 non-null object
dtypes: float64(1), int64(8), object(5)
memory usage: 11.8+ MB

    \end{Verbatim}

    \begin{Verbatim}[commandchars=\\\{\}]
{\color{incolor}In [{\color{incolor}5}]:} \PY{c+c1}{\PYZsh{} The following command will create a sub\PYZhy{}group of the appt dataframe, where the column \PYZsq{}No\PYZhy{}show\PYZsq{} value is \PYZsq{}Yes\PYZsq{}.}
        \PY{c+c1}{\PYZsh{} Meaning, this sub\PYZhy{}group of the appt dataframe is all the missed or \PYZsq{}no\PYZhy{}show\PYZsq{} appointments.}
        \PY{c+c1}{\PYZsh{} This could be good later for exploring differences between those who showed up for appointments or didn\PYZsq{}t.}
        \PY{n}{appt\PYZus{}yes} \PY{o}{=} \PY{n}{appt}\PY{p}{[}\PY{n}{appt}\PY{p}{[}\PY{l+s+s1}{\PYZsq{}}\PY{l+s+s1}{No\PYZhy{}show}\PY{l+s+s1}{\PYZsq{}}\PY{p}{]} \PY{o}{==} \PY{l+s+s1}{\PYZsq{}}\PY{l+s+s1}{Yes}\PY{l+s+s1}{\PYZsq{}}\PY{p}{]}
        
        \PY{c+c1}{\PYZsh{} Perform operations to inspect data types and look for instances of missing or possibly errant data.}
        \PY{c+c1}{\PYZsh{} Again, using the describe function allows us to see the statistics for this \PYZsq{}no\PYZhy{}show\PYZsq{} sub\PYZhy{}group from \PYZsq{}appt\PYZsq{}.}
        \PY{n}{appt\PYZus{}yes}\PY{o}{.}\PY{n}{describe}\PY{p}{(}\PY{p}{)}
\end{Verbatim}


\begin{Verbatim}[commandchars=\\\{\}]
{\color{outcolor}Out[{\color{outcolor}5}]:}           PatientId  AppointmentID           Age   Scholarship  Hipertension  \textbackslash{}
        count  2.231900e+04   2.231900e+04  22319.000000  22319.000000  22319.000000   
        mean   1.467530e+14   5.652259e+06     34.317667      0.115507      0.169004   
        std    2.549922e+14   7.429686e+04     21.965941      0.319640      0.374764   
        min    5.628261e+06   5.122866e+06      0.000000      0.000000      0.000000   
        25\%    4.180000e+12   5.614192e+06     16.000000      0.000000      0.000000   
        50\%    3.160000e+13   5.657916e+06     33.000000      0.000000      0.000000   
        75\%    9.450000e+13   5.703175e+06     51.000000      0.000000      0.000000   
        max    1.000000e+15   5.789986e+06    115.000000      1.000000      1.000000   
        
                   Diabetes    Alcoholism       Handcap  SMS\_received  
        count  22319.000000  22319.000000  22319.000000  22319.000000  
        mean       0.064071      0.030333      0.020297      0.438371  
        std        0.244885      0.171505      0.156670      0.496198  
        min        0.000000      0.000000      0.000000      0.000000  
        25\%        0.000000      0.000000      0.000000      0.000000  
        50\%        0.000000      0.000000      0.000000      0.000000  
        75\%        0.000000      0.000000      0.000000      1.000000  
        max        1.000000      1.000000      4.000000      1.000000  
\end{Verbatim}
            
    \begin{Verbatim}[commandchars=\\\{\}]
{\color{incolor}In [{\color{incolor}6}]:} \PY{c+c1}{\PYZsh{} Perform operations to inspect data types and look for instances of missing or possibly errant data.}
        \PY{c+c1}{\PYZsh{} Again, using info we can confirm that there are still a uniform number of rows for each column name.}
        \PY{n}{appt\PYZus{}yes}\PY{o}{.}\PY{n}{info}\PY{p}{(}\PY{p}{)}
\end{Verbatim}


    \begin{Verbatim}[commandchars=\\\{\}]
<class 'pandas.core.frame.DataFrame'>
Int64Index: 22319 entries, 6 to 110516
Data columns (total 14 columns):
PatientId         22319 non-null float64
AppointmentID     22319 non-null int64
Gender            22319 non-null object
ScheduledDay      22319 non-null object
AppointmentDay    22319 non-null object
Age               22319 non-null int64
Neighbourhood     22319 non-null object
Scholarship       22319 non-null int64
Hipertension      22319 non-null int64
Diabetes          22319 non-null int64
Alcoholism        22319 non-null int64
Handcap           22319 non-null int64
SMS\_received      22319 non-null int64
No-show           22319 non-null object
dtypes: float64(1), int64(8), object(5)
memory usage: 2.6+ MB

    \end{Verbatim}

    \begin{Verbatim}[commandchars=\\\{\}]
{\color{incolor}In [{\color{incolor}7}]:} \PY{c+c1}{\PYZsh{} The following command will create a sub\PYZhy{}group of the appt dataframe, where the column \PYZsq{}No\PYZhy{}show\PYZsq{} value is \PYZsq{}No\PYZsq{}.}
        \PY{c+c1}{\PYZsh{} Meaning, this sub\PYZhy{}group of the appt dataframe is all the arrived or people who did show up for appointments.}
        \PY{c+c1}{\PYZsh{} This could be good later for exploring differences between those who showed up for appointments or didn\PYZsq{}t.}
        \PY{n}{appt\PYZus{}no} \PY{o}{=} \PY{n}{appt}\PY{p}{[}\PY{n}{appt}\PY{p}{[}\PY{l+s+s1}{\PYZsq{}}\PY{l+s+s1}{No\PYZhy{}show}\PY{l+s+s1}{\PYZsq{}}\PY{p}{]} \PY{o}{==} \PY{l+s+s1}{\PYZsq{}}\PY{l+s+s1}{No}\PY{l+s+s1}{\PYZsq{}}\PY{p}{]}
        
        \PY{c+c1}{\PYZsh{} Perform operations to inspect data types and look for instances of missing or possibly errant data.}
        \PY{c+c1}{\PYZsh{} Again, using the describe function allows us to see the statistics for this \PYZsq{}did\PYZhy{}show\PYZsq{} sub\PYZhy{}group from \PYZsq{}appt\PYZsq{}.}
        \PY{n}{appt\PYZus{}no}\PY{o}{.}\PY{n}{describe}\PY{p}{(}\PY{p}{)}
\end{Verbatim}


\begin{Verbatim}[commandchars=\\\{\}]
{\color{outcolor}Out[{\color{outcolor}7}]:}           PatientId  AppointmentID           Age   Scholarship  Hipertension  \textbackslash{}
        count  8.820800e+04   8.820800e+04  88208.000000  88208.000000  88208.000000   
        mean   1.476841e+14   5.681137e+06     37.790064      0.093903      0.204392   
        std    2.563736e+14   6.931225e+04     23.338878      0.291695      0.403259   
        min    3.920000e+04   5.030230e+06     -1.000000      0.000000      0.000000   
        25\%    4.170000e+12   5.646218e+06     18.000000      0.000000      0.000000   
        50\%    3.180000e+13   5.685684e+06     38.000000      0.000000      0.000000   
        75\%    9.430000e+13   5.731078e+06     56.000000      0.000000      0.000000   
        max    1.000000e+15   5.790484e+06    115.000000      1.000000      1.000000   
        
                   Diabetes    Alcoholism       Handcap  SMS\_received  
        count  88208.000000  88208.000000  88208.000000  88208.000000  
        mean       0.073837      0.030417      0.022742      0.291334  
        std        0.261507      0.171732      0.162750      0.454380  
        min        0.000000      0.000000      0.000000      0.000000  
        25\%        0.000000      0.000000      0.000000      0.000000  
        50\%        0.000000      0.000000      0.000000      0.000000  
        75\%        0.000000      0.000000      0.000000      1.000000  
        max        1.000000      1.000000      4.000000      1.000000  
\end{Verbatim}
            
    \begin{Verbatim}[commandchars=\\\{\}]
{\color{incolor}In [{\color{incolor}8}]:} \PY{c+c1}{\PYZsh{} Perform operations to inspect data types and look for instances of missing or possibly errant data.}
        \PY{c+c1}{\PYZsh{} Again, using info we can confirm that there are still a uniform number of rows for each column name.}
        \PY{n}{appt\PYZus{}no}\PY{o}{.}\PY{n}{info}\PY{p}{(}\PY{p}{)}
\end{Verbatim}


    \begin{Verbatim}[commandchars=\\\{\}]
<class 'pandas.core.frame.DataFrame'>
Int64Index: 88208 entries, 0 to 110526
Data columns (total 14 columns):
PatientId         88208 non-null float64
AppointmentID     88208 non-null int64
Gender            88208 non-null object
ScheduledDay      88208 non-null object
AppointmentDay    88208 non-null object
Age               88208 non-null int64
Neighbourhood     88208 non-null object
Scholarship       88208 non-null int64
Hipertension      88208 non-null int64
Diabetes          88208 non-null int64
Alcoholism        88208 non-null int64
Handcap           88208 non-null int64
SMS\_received      88208 non-null int64
No-show           88208 non-null object
dtypes: float64(1), int64(8), object(5)
memory usage: 10.1+ MB

    \end{Verbatim}

    \subsubsection{Data Cleaning}\label{data-cleaning}

    \begin{Verbatim}[commandchars=\\\{\}]
{\color{incolor}In [{\color{incolor}9}]:} \PY{c+c1}{\PYZsh{} Taking a closer look at some aspects of the data from the above dataframes.}
        \PY{c+c1}{\PYZsh{} The following few cells of code in this section are looking into the maximum listed age of 115}
        \PY{c+c1}{\PYZsh{} First create a sub\PYZhy{}group of \PYZsq{}appt\PYZsq{} dataframe where \PYZsq{}Age\PYZsq{} column values are equal to \PYZsq{}115\PYZsq{}}
        \PY{n}{max\PYZus{}age} \PY{o}{=} \PY{n}{appt}\PY{p}{[}\PY{n}{appt}\PY{p}{[}\PY{l+s+s1}{\PYZsq{}}\PY{l+s+s1}{Age}\PY{l+s+s1}{\PYZsq{}}\PY{p}{]} \PY{o}{==} \PY{l+m+mi}{115}\PY{p}{]}
        \PY{n}{max\PYZus{}age}
\end{Verbatim}


\begin{Verbatim}[commandchars=\\\{\}]
{\color{outcolor}Out[{\color{outcolor}9}]:}           PatientId  AppointmentID Gender          ScheduledDay  \textbackslash{}
        63912  3.200000e+13        5700278      F  2016-05-16T09:17:44Z   
        63915  3.200000e+13        5700279      F  2016-05-16T09:17:44Z   
        68127  3.200000e+13        5562812      F  2016-04-08T14:29:17Z   
        76284  3.200000e+13        5744037      F  2016-05-30T09:44:51Z   
        97666  7.480000e+14        5717451      F  2016-05-19T07:57:56Z   
        
                     AppointmentDay  Age Neighbourhood  Scholarship  Hipertension  \textbackslash{}
        63912  2016-05-19T00:00:00Z  115    ANDORINHAS            0             0   
        63915  2016-05-19T00:00:00Z  115    ANDORINHAS            0             0   
        68127  2016-05-16T00:00:00Z  115    ANDORINHAS            0             0   
        76284  2016-05-30T00:00:00Z  115    ANDORINHAS            0             0   
        97666  2016-06-03T00:00:00Z  115    Sí€O JOSíŠ            0             1   
        
               Diabetes  Alcoholism  Handcap  SMS\_received No-show  
        63912         0           0        1             0     Yes  
        63915         0           0        1             0     Yes  
        68127         0           0        1             0     Yes  
        76284         0           0        1             0      No  
        97666         0           0        0             1      No  
\end{Verbatim}
            
    \begin{Verbatim}[commandchars=\\\{\}]
{\color{incolor}In [{\color{incolor}10}]:} \PY{c+c1}{\PYZsh{} The following counts the number of unique values in a column (PatientId); we see 2 people are 115 years old.}
         \PY{c+c1}{\PYZsh{} Given the wide range of ages in the dataset it isn\PYZsq{}t possible to tell if these are errors or correct age values.}
         \PY{n}{max\PYZus{}age}\PY{p}{[}\PY{l+s+s1}{\PYZsq{}}\PY{l+s+s1}{PatientId}\PY{l+s+s1}{\PYZsq{}}\PY{p}{]}\PY{o}{.}\PY{n}{nunique}\PY{p}{(}\PY{p}{)}
\end{Verbatim}


\begin{Verbatim}[commandchars=\\\{\}]
{\color{outcolor}Out[{\color{outcolor}10}]:} 2
\end{Verbatim}
            
    \begin{Verbatim}[commandchars=\\\{\}]
{\color{incolor}In [{\color{incolor}11}]:} \PY{c+c1}{\PYZsh{} Before moving onto the research questions we should clean up the age error of \PYZhy{}1 as is show in \PYZsq{}appt\PYZus{}no\PYZsq{}.}
         \PY{c+c1}{\PYZsh{} Clearly people can\PYZsq{}t have an age of \PYZhy{}1 as that would be pre\PYZhy{}birth and at birth you are essentially 0 years old.}
         \PY{c+c1}{\PYZsh{} We wil go ahead and select this patient and change their age to 0 and assume this is a new born.}
         \PY{c+c1}{\PYZsh{} This will use basically the same code as exploring the 115 year old people above.}
         \PY{n}{min\PYZus{}age} \PY{o}{=} \PY{n}{appt}\PY{p}{[}\PY{n}{appt}\PY{p}{[}\PY{l+s+s1}{\PYZsq{}}\PY{l+s+s1}{Age}\PY{l+s+s1}{\PYZsq{}}\PY{p}{]} \PY{o}{==} \PY{o}{\PYZhy{}}\PY{l+m+mi}{1}\PY{p}{]}
         
         \PY{c+c1}{\PYZsh{} This line of code replaces, in the Age column, the value(s) that are \PYZhy{}1 and changes them to 0 (inplace).}
         \PY{n}{appt}\PY{o}{.}\PY{n}{Age}\PY{o}{.}\PY{n}{replace}\PY{p}{(}\PY{p}{[}\PY{o}{\PYZhy{}}\PY{l+m+mi}{1}\PY{p}{]}\PY{p}{,}\PY{p}{[}\PY{l+m+mi}{0}\PY{p}{]}\PY{p}{,} \PY{n}{inplace} \PY{o}{=} \PY{k+kc}{True}\PY{p}{)}
\end{Verbatim}


    \begin{Verbatim}[commandchars=\\\{\}]
{\color{incolor}In [{\color{incolor}12}]:} \PY{c+c1}{\PYZsh{} To show that it worked and the row wasn\PYZsq{}t deleted, we will re\PYZhy{}run some basic \PYZsq{}tests\PYZsq{}.}
         \PY{c+c1}{\PYZsh{} We can compare the results to the previous \PYZsq{}tests\PYZsq{} on the \PYZsq{}appt\PYZsq{} or \PYZsq{}appt\PYZus{}no\PYZsq{} dataframe from above.}
         \PY{c+c1}{\PYZsh{} We can see that mininum Age is now only 0 and no longer \PYZhy{}1. This will slightly alter the other statistics.}
         \PY{n}{appt}\PY{o}{.}\PY{n}{describe}\PY{p}{(}\PY{p}{)}
\end{Verbatim}


\begin{Verbatim}[commandchars=\\\{\}]
{\color{outcolor}Out[{\color{outcolor}12}]:}           PatientId  AppointmentID            Age    Scholarship  \textbackslash{}
         count  1.105270e+05   1.105270e+05  110527.000000  110527.000000   
         mean   1.474961e+14   5.675305e+06      37.088883       0.098266   
         std    2.560943e+14   7.129575e+04      23.110190       0.297675   
         min    3.920000e+04   5.030230e+06       0.000000       0.000000   
         25\%    4.170000e+12   5.640286e+06      18.000000       0.000000   
         50\%    3.170000e+13   5.680573e+06      37.000000       0.000000   
         75\%    9.440000e+13   5.725524e+06      55.000000       0.000000   
         max    1.000000e+15   5.790484e+06     115.000000       1.000000   
         
                 Hipertension       Diabetes     Alcoholism        Handcap  \textbackslash{}
         count  110527.000000  110527.000000  110527.000000  110527.000000   
         mean        0.197246       0.071865       0.030400       0.022248   
         std         0.397921       0.258265       0.171686       0.161543   
         min         0.000000       0.000000       0.000000       0.000000   
         25\%         0.000000       0.000000       0.000000       0.000000   
         50\%         0.000000       0.000000       0.000000       0.000000   
         75\%         0.000000       0.000000       0.000000       0.000000   
         max         1.000000       1.000000       1.000000       4.000000   
         
                 SMS\_received  
         count  110527.000000  
         mean        0.321026  
         std         0.466873  
         min         0.000000  
         25\%         0.000000  
         50\%         0.000000  
         75\%         1.000000  
         max         1.000000  
\end{Verbatim}
            
    \begin{Verbatim}[commandchars=\\\{\}]
{\color{incolor}In [{\color{incolor}13}]:} \PY{c+c1}{\PYZsh{} Anoter \PYZsq{}test\PYZsq{} for quick comparison purposes to make sure the Age \PYZhy{}1 replacement to 0 worked correctly.}
         \PY{c+c1}{\PYZsh{} There\PYZsq{}s still 110,527 rows which shows that the row with Age \PYZhy{}1 wasn\PYZsq{}t removed but the \PYZhy{}1 changed to 0.}
         \PY{n}{appt}\PY{o}{.}\PY{n}{info}\PY{p}{(}\PY{p}{)}
\end{Verbatim}


    \begin{Verbatim}[commandchars=\\\{\}]
<class 'pandas.core.frame.DataFrame'>
RangeIndex: 110527 entries, 0 to 110526
Data columns (total 14 columns):
PatientId         110527 non-null float64
AppointmentID     110527 non-null int64
Gender            110527 non-null object
ScheduledDay      110527 non-null object
AppointmentDay    110527 non-null object
Age               110527 non-null int64
Neighbourhood     110527 non-null object
Scholarship       110527 non-null int64
Hipertension      110527 non-null int64
Diabetes          110527 non-null int64
Alcoholism        110527 non-null int64
Handcap           110527 non-null int64
SMS\_received      110527 non-null int64
No-show           110527 non-null object
dtypes: float64(1), int64(8), object(5)
memory usage: 11.8+ MB

    \end{Verbatim}

     \#\# Exploratory Data Analysis

\subsubsection{Research Question 1 (single variable
exploration):}\label{research-question-1-single-variable-exploration}

\subsubsection{Does age (younger or older) have any kind of impact or
relationship in terms of people showing up for appointments or
not?}\label{does-age-younger-or-older-have-any-kind-of-impact-or-relationship-in-terms-of-people-showing-up-for-appointments-or-not}

    \begin{Verbatim}[commandchars=\\\{\}]
{\color{incolor}In [{\color{incolor}14}]:} \PY{c+c1}{\PYZsh{} In order to explore the above question we save a copy of the \PYZsq{}appt\PYZsq{} dataframe to the \PYZsq{}new\PYZus{}appt\PYZsq{} name.}
         \PY{n}{new\PYZus{}appt} \PY{o}{=} \PY{n}{appt}
         
         \PY{c+c1}{\PYZsh{} Now, using the \PYZsq{}new\PYZus{}appt\PYZsq{} dataframe, we create a boxplot and use Age as the x\PYZhy{}axis and \PYZdq{}No\PYZhy{}show\PYZdq{} for the y\PYZhy{}axis.}
         \PY{c+c1}{\PYZsh{} palette=[\PYZdq{}g\PYZdq{}, \PYZdq{}r\PYZdq{}] sets the colors for the 2 categories, Yes and No. Yes (missed appointment); No is opposite.}
         \PY{n}{appt\PYZus{}yes\PYZus{}viz} \PY{o}{=} \PY{n}{sns}\PY{o}{.}\PY{n}{boxplot}\PY{p}{(}\PY{n}{x}\PY{o}{=}\PY{l+s+s2}{\PYZdq{}}\PY{l+s+s2}{Age}\PY{l+s+s2}{\PYZdq{}}\PY{p}{,} \PY{n}{y}\PY{o}{=}\PY{l+s+s2}{\PYZdq{}}\PY{l+s+s2}{No\PYZhy{}show}\PY{l+s+s2}{\PYZdq{}}\PY{p}{,} \PY{n}{palette}\PY{o}{=}\PY{p}{[}\PY{l+s+s2}{\PYZdq{}}\PY{l+s+s2}{g}\PY{l+s+s2}{\PYZdq{}}\PY{p}{,} \PY{l+s+s2}{\PYZdq{}}\PY{l+s+s2}{r}\PY{l+s+s2}{\PYZdq{}}\PY{p}{]}\PY{p}{,} \PY{n}{data}\PY{o}{=}\PY{n}{new\PYZus{}appt}\PY{p}{)}
\end{Verbatim}


    \begin{center}
    \adjustimage{max size={0.9\linewidth}{0.9\paperheight}}{output_18_0.png}
    \end{center}
    { \hspace*{\fill} \\}
    
    \subsubsection{Research Question 2 (multi-variable
exploration):}\label{research-question-2-multi-variable-exploration}

\subsubsection{Does the combination of age and scholarship have some
kind of impact or relationship in terms of people showing up for
appointments or
not?}\label{does-the-combination-of-age-and-scholarship-have-some-kind-of-impact-or-relationship-in-terms-of-people-showing-up-for-appointments-or-not}

    \begin{Verbatim}[commandchars=\\\{\}]
{\color{incolor}In [{\color{incolor}15}]:} \PY{c+c1}{\PYZsh{} Draw a barplot to show \PYZdq{}No\PYZhy{}show\PYZdq{} status (x\PYZhy{}axis) and \PYZdq{}Age\PYZdq{} (y\PYZhy{}axis) and \PYZdq{}Scholarship\PYZdq{} as the hue.}
         \PY{c+c1}{\PYZsh{} For the \PYZdq{}Scholarship\PYZdq{} hue, the blue (0) means those people didn\PYZsq{}t have healthcare scholarships but means yes.}
         \PY{n}{sns}\PY{o}{.}\PY{n}{catplot}\PY{p}{(}\PY{n}{x}\PY{o}{=}\PY{l+s+s2}{\PYZdq{}}\PY{l+s+s2}{No\PYZhy{}show}\PY{l+s+s2}{\PYZdq{}}\PY{p}{,} \PY{n}{y}\PY{o}{=}\PY{l+s+s2}{\PYZdq{}}\PY{l+s+s2}{Age}\PY{l+s+s2}{\PYZdq{}}\PY{p}{,} \PY{n}{hue}\PY{o}{=}\PY{l+s+s2}{\PYZdq{}}\PY{l+s+s2}{Scholarship}\PY{l+s+s2}{\PYZdq{}}\PY{p}{,} \PY{n}{data}\PY{o}{=}\PY{n}{new\PYZus{}appt}\PY{p}{,} \PY{n}{height}\PY{o}{=}\PY{l+m+mi}{6}\PY{p}{,} \PY{n}{kind}\PY{o}{=}\PY{l+s+s2}{\PYZdq{}}\PY{l+s+s2}{bar}\PY{l+s+s2}{\PYZdq{}}\PY{p}{,} \PY{n}{palette}\PY{o}{=}\PY{l+s+s2}{\PYZdq{}}\PY{l+s+s2}{muted}\PY{l+s+s2}{\PYZdq{}}\PY{p}{)}
\end{Verbatim}


\begin{Verbatim}[commandchars=\\\{\}]
{\color{outcolor}Out[{\color{outcolor}15}]:} <seaborn.axisgrid.FacetGrid at 0x1a169c1470>
\end{Verbatim}
            
    \begin{center}
    \adjustimage{max size={0.9\linewidth}{0.9\paperheight}}{output_20_1.png}
    \end{center}
    { \hspace*{\fill} \\}
    
     \#\# Conclusions \textgreater{} Question 1 results and conclusions
based on the boxplot: \textgreater{} \textgreater{} Right away we can
see from the boxplot that the people who did not show up for their
appointments were younger. \textgreater{} This doesn't account for the
fact that the number of people in "Yes" is about four times smaller than
"No". \textgreater{} I imagine that if each group had closer or same
numbers that the "Yes" and "No" visuals would be more similar.
\textgreater{} Further analysis could be done to equalize the "Yes" and
"No" group numbers and then re-make the boxplot. \textgreater{}
\textgreater{} Question 2 results and conclusions based on the barplot:
\textgreater{} \textgreater{} From the barplot we can see that in the
'didn't show up for an appointment' group (Yes) that the people
\textgreater{} who don't have scholarships (0) were closer in age to
those in the same group but did have scholarships (1). \textgreater{} In
the 'did show up for an appointment' group (No), the age difference is
noticeably larger, so much larger \textgreater{} in fact that the people
who do have scholarships in the "No" group were almost as young as the
people with \textgreater{} scholarships in the "Yes" group. This is
interesting because (as previously mentioned) the "No" group has
\textgreater{} nearly four times more people than the "Yes" group.
\textgreater{} \textgreater{} Overall, after exploring and cleaning this
dataset - I don't believe there is definitive proof that either Age
\textgreater{} nor Scholarship status has an impact on whether or not a
person shows up for their appointment or not. This is \textgreater{}
partly due to the fact that the number of people who did show up for
appointments was nearly four times more \textgreater{} than the number
of people who didn't show up for appointments. Further work could be
done to balance out the \textgreater{} numbers from the "Yes" and "No"
groups to see if the statistical data and visuals in this project become
more \textgreater{} closely aligned; it would be an anomaly for the data
to be in such a way that the statistics and visuals for \textgreater{}
each group were to be the (exact) same.


    % Add a bibliography block to the postdoc
    
    
    
    \end{document}
